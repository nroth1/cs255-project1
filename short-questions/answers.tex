\documentclass[12pt]{article}
\usepackage[margin=.75in]{geometry} 
\usepackage{amsmath,amsthm,amssymb}
\usepackage{graphicx}
\usepackage{upgreek}
\newcommand{\N}{\mathbb{N}}
\newcommand{\Z}{\mathbb{Z}}
\newenvironment{theorem}[2][Theorem]{\begin{trivlist}
\item[\hskip \labelsep {\bfseries #1}\hskip \labelsep {\bfseries #2.}]}{\end{trivlist}}
\newenvironment{lemma}[2][Lemma]{\begin{trivlist}
\item[\hskip \labelsep {\bfseries #1}\hskip \labelsep {\bfseries #2.}]}{\end{trivlist}}
\newenvironment{exercise}[2][Exercise]{\begin{trivlist}
\item[\hskip \labelsep {\bfseries #1}\hskip \labelsep {\bfseries #2.}]}{\end{trivlist}}
\newenvironment{problem}[2][Problem]{\begin{trivlist}
\item[\hskip \labelsep {\bfseries #1}\hskip \labelsep {\bfseries #2.}]}{\end{trivlist}}
\newenvironment{question}[2][Question]{\begin{trivlist}
\item[\hskip \labelsep {\bfseries #1}\hskip \labelsep {\bfseries #2.}]}{\end{trivlist}}
\newenvironment{corollary}[2][Corollary]{\begin{trivlist}
\item[\hskip \labelsep {\bfseries #1}\hskip \labelsep {\bfseries #2.}]}{\end{trivlist}}


\usepackage{lipsum} % for filler text
\usepackage{fancyhdr}
\pagestyle{fancy}
\fancyhead{} % clear all header fields
\renewcommand{\headrulewidth}{0pt} % no line in header area
\fancyfoot{} % clear all footer fields

\begin{document}

% --------------------------------------------------------------
%                          Title
% --------------------------------------------------------------
\title{Project 1}
\author{Yanshu Hong, Nat Roth\\ 
CS255: Cryptography and Computer Security} 

\maketitle 
% --------------------------------------------------------------
%                          Main
% --------------------------------------------------------------
 

\begin{question}{1}
% Briefly describe your method for preventing the adversary from learning information about the lengths of the passwords stored in your password manager.


\end{question}



\begin{question}{2}
% Briefly describe your method for preventing swap attacks (Section 2.2). Provide an argument for why the attack is prevented in your scheme.


\end{question}



\begin{question}{3} 
% In our proposed defense against the rollback attack (Section 2.2), we assume that we can store the SHA-256 hash in a trusted location beyond the reach of an adversary. Is it necessary to assume that such a trusted location exists, in order to defend against rollback attacks? Briefly justify your answer.


\end{question}



\begin{question}{4} 
% What if we had used a different MAC (other than HMAC) on the domain names to produce the keys for the key-value store? Would the scheme still satisfy the desired security properties? Either show this, or give an example of a secure MAC for which the resulting password manager implementation would be insecure.


\end{question}



\begin{question}{5} 
% In our specification, we leak the number of records in the password manager. Describe an approach to reduce or completely eliminate the information leaked about the number of records.


\end{question}

 
% --------------------------------------------------------------
%     You don't have to mess with anything below this line.
% --------------------------------------------------------------
 
\end{document}

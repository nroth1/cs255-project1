\documentclass[12pt]{article}
\usepackage[margin=.75in]{geometry} 
\usepackage{amsmath,amsthm,amssymb}
\usepackage{graphicx}
\usepackage{upgreek}
\newcommand{\N}{\mathbb{N}}
\newcommand{\Z}{\mathbb{Z}}
\newenvironment{theorem}[2][Theorem]{\begin{trivlist}
\item[\hskip \labelsep {\bfseries #1}\hskip \labelsep {\bfseries #2.}]}{\end{trivlist}}
\newenvironment{lemma}[2][Lemma]{\begin{trivlist}
\item[\hskip \labelsep {\bfseries #1}\hskip \labelsep {\bfseries #2.}]}{\end{trivlist}}
\newenvironment{exercise}[2][Exercise]{\begin{trivlist}
\item[\hskip \labelsep {\bfseries #1}\hskip \labelsep {\bfseries #2.}]}{\end{trivlist}}
\newenvironment{problem}[2][Problem]{\begin{trivlist}
\item[\hskip \labelsep {\bfseries #1}\hskip \labelsep {\bfseries #2.}]}{\end{trivlist}}
\newenvironment{question}[2][Question]{\begin{trivlist}
\item[\hskip \labelsep {\bfseries #1}\hskip \labelsep {\bfseries #2.}]}{\end{trivlist}}
\newenvironment{corollary}[2][Corollary]{\begin{trivlist}
\item[\hskip \labelsep {\bfseries #1}\hskip \labelsep {\bfseries #2.}]}{\end{trivlist}}


\usepackage{lipsum} % for filler text
\usepackage{fancyhdr}
\pagestyle{fancy}
\fancyhead{} % clear all header fields
\renewcommand{\headrulewidth}{0pt} % no line in header area
\fancyfoot{} % clear all footer fields

\begin{document}

% --------------------------------------------------------------
%                          Title
% --------------------------------------------------------------
\title{Project 1}
\author{Yanshu Hong, Nat Roth\\ 
CS255: Cryptography and Computer Security} 

\maketitle 
% --------------------------------------------------------------
%                          Main
% --------------------------------------------------------------
 

\begin{question}{1}
% Briefly describe your method for preventing the adversary from learning information about the lengths of the passwords stored in your password manager.
All the passwords are padded to 65 bytes (one more byte than the largest possible password allowed) before being encrypted. This means that all passwords have the same length and thus the encryption does not leak the lengths of the passwords.

\end{question}



\begin{question}{2}
% Briefly describe your method for preventing swap attacks (Section 2.2). Provide an argument for why the attack is prevented in your scheme.
Each padded password is concatenated with the HMAC of its domain before being encrypted. The value in the KVS is the encryption of this concatenation. Every time the user requests the password for a domain, the value is decrypted and split into padded password and HMAC (because the padded password has a constant length, the splitting point is known). Then this HMAC will be compared with the HAMC of the domain the user passed in. If they're different, an exception will be thrown.

If a swap attack occurs, then the HMAC of the key will not match the HMAC encrypted inside the value. Therefore, the GET() function can detect any swap.  

\end{question}



\begin{question}{3} 
% In our proposed defense against the rollback attack (Section 2.2), we assume that we can store the SHA-256 hash in a trusted location beyond the reach of an adversary. Is it necessary to assume that such a trusted location exists, in order to defend against rollback attacks? Briefly justify your answer.
Yes, the trusted location must exist in order to defend rollback attacks. 

Suppose the keychain is secure against rollbacks but there is no trusted location. All data is stored on the vulnerable disk which is accessible to the attacker. Say the attacker wants to perform a rollback attack, he could save the disk image $D_1$ at time $t_1$. Later on at time $t_2$, the attacker could replace the entire disk image by $D_1$ and the entire keychain could be rolled back to the state at time $t_1$. Because all data (including SHA-256 or data of any other authentication scheme) the keychain relies on for authentication must be saved on the vulnerable disk, if the keychain is able to detect the rollback attack when reading from the tampered disk image after $t_2$, it should throw out a similar exception at time $t_1$ because it is facing exactly the same disk image at $t_1$ and $t_2$. This is a contradiction because we had a valid state at $t_1$ and throwing out an exception at $t_1$ was impossible. Therefore, there must be a trusted location in order to be secure against rollback attacks.

If the trusted location exists, some data can be stored there. Therefore, the attacker can not completely restore the keychain state at time $t_1$ using the disk image $D_1$.

\end{question}



\begin{question}{4} 
% What if we had used a different MAC (other than HMAC) on the domain names to produce the keys for the key-value store? Would the scheme still satisfy the desired security properties? Either show this, or give an example of a secure MAC for which the resulting password manager implementation would be insecure.
In general, MAC might not be collision resistant. Therefore, we might be possible to find two different domains $d_1$ and $d_2$ that map to the same tag under the MAC with non-negligible probability. Once $d_1$ and $d_2$ are found, the attacker can perform the following attack:

1) specify password $p_0$ or $p_0$ to be added under domain $d_1$

2) specify password $p_0$ or $p_1$ ($p_0 \not= p_1$) to be added under domain $d_2$

3) specify domain $d_1$ for authentication and receive the password $p'$

4) compare $p'$ to $p_0$ and $p_1$ and output the world bit

The attacker can perform step 3 because step 1 is the last query to domain $d_1$ and the attacker sent in identical passwords. Because $d_1$ and $d_2$ map to the same HMAC, the value of domain $d_1$ will be overwritten in step 2, namely, by either the encryption of $p_0$ or the encryption of $p_1$. Therefore, the attacker can find out the world bit at step 4 with non-negligible probability.

HMAC is implemented by SHA-256 and it should be collision resistant. However, another secure MAC might be not collision resistant and therefore might result the password manager to be insecure.

\end{question}



\begin{question}{5} 
% In our specification, we leak the number of records in the password manager. Describe an approach to reduce or completely eliminate the information leaked about the number of records.
If we can assume some reasonable upper-bound for the number of records, we can always add dummy records to the KVS so that all KVSs will have the same number of entries. In general, people will have not have a lot passwords. The reasonable upper-bound might be pretty small, potentially in the thousands. And this should not be too memory-intensive. 

To construct dummy records, we just choose random strings as domains and random strings as passwords. The possibility that a randomly chosen domain will be mapped to a user chosen one under HMAC is negligible, assuming the user has reasonable number of passwords.

\end{question}

 
% --------------------------------------------------------------
%     You don't have to mess with anything below this line.
% --------------------------------------------------------------
 
\end{document}
